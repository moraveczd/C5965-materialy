\documentclass[hyperref=unicode,presentation,10pt]{beamer}

\usepackage[absolute,overlay]{textpos}
\usepackage{array}
\usepackage{graphicx}
\usepackage{adjustbox}
\usepackage[version=4]{mhchem}
\usepackage{chemfig}
\usepackage{caption}

%dělení slov
\usepackage{ragged2e}
\let\raggedright=\RaggedRight
%konec dělení slov

\addtobeamertemplate{frametitle}{
	\let\insertframetitle\insertsectionhead}{}
\addtobeamertemplate{frametitle}{
	\let\insertframesubtitle\insertsubsectionhead}{}

\makeatletter
\CheckCommand*\beamer@checkframetitle{\@ifnextchar\bgroup\beamer@inlineframetitle{}}
\renewcommand*\beamer@checkframetitle{\global\let\beamer@frametitle\relax\@ifnextchar\bgroup\beamer@inlineframetitle{}}
\makeatother
\setbeamercolor{section in toc}{fg=red}
\setbeamertemplate{section in toc shaded}[default][100]

\usepackage{fontspec}
\usepackage{unicode-math}

\usepackage{polyglossia}
\setdefaultlanguage{czech}

\def\uv#1{„#1“}

\mode<presentation>{\usetheme{default}}
\usecolortheme{crane}

\setbeamertemplate{footline}[frame number]

\title[Crisis]
{Termická analýza}

\author
{Zdeněk Moravec}
\date[KPT 2004]
{hugo@chemi.muni.cz}

\begin{document}

\frame{\titlepage}

\section{Termická analýza}
\frame{
	\frametitle{}
	\vfill
	\begin{itemize}
	\item Soubor metod sledujících chování vzorku během definovaného teplotního programu
	\item TG - termogravimetrie - sledujeme změny hmotnosti vzorku
	\item DTA - diferenční termická analýza - sledujeme rozdíl teplot mezi vzorkem a referencí během teplotního programu
	\item DSC - diferenční skenovací kalorimetrie - měříme tepelný tok mezi vzorkem a referencí
	\item STA - simultánní termická analýza
	\item TMA - Termomechanická analýza - sledujeme deformace zatíženého vzorku během teplotního programu
	\end{itemize}
	\vfill
}

\subsection{Termogravimetrie}
\frame{
	\frametitle{}
	\vfill
	\begin{itemize}
	\item Měříme změny hmotnosti vzorku při jeho plynulém ohřevu nebo ochlazování
	\item Změny hmotnosti se vyjadřují jako závislost na teplotě nebo čase analýzy
	\item  Během ohřevu může docházet k poklesu hmotnosti vzorku, z~důvodu uvolňování plynu, např. vody nebo oxidu uhličitého
	\item Může také docházet ke zvyšování hmotnosti vzorku reakcí s~atmosférou - oxidace vzorku
	\item TG křivky podávají informace o složení zkoumaného vzorku, jeho tepelné stálosti, teplotním rozkladu a také o produktech vznikajících při rozkladu
	\item TG křivka ve svém průběhu obsahuje úseky vodorovné s osou x, tzv. prodlevy, a~zlomy.
	\begin{itemize}
		\item Prodlevy jsou úseky, kdy ještě nedošlo k žádné změně hmotnosti vzorku.
		\item Zlomy naopak naznačují, že se analyzovaný vzorek začíná rozkládat (mění svoji hmotnost).
	\end{itemize}
	\end{itemize}
	\vfill
}

\frame{
	\frametitle{}
	\vfill
	\begin{figure}
		\adjincludegraphics[height=.7\textheight]{img/tg-skalice.png}
		\caption*{TG křivka modré skalice.}
	\end{figure}
	\vfill
}

\subsection{DTA}
\frame{
	\frametitle{}
	\begin{columns}
		\begin{column}{0.5\textwidth}
	\vfill
	\begin{itemize}
	\item Diferenční Termická Analýza
	\item Předchůdce DSC, jednodušší instrumentace
	\item Umožňuje měřit tepelné změny vzorku během analýzy
	\item Měříme rozdíl teploty kelímku se vzorkem a referenčního kelímku
	\end{itemize}
	\vfill
		\end{column}
		\begin{column}{0.5\textwidth}
			\begin{figure}
				\adjincludegraphics[width=\textwidth]{img/DTA.jpg}
				\caption*{DTA držák s korundovými kelímky.}
			\end{figure}
		\end{column}
	\end{columns}
}

\subsection{DSC}
\frame{
	\frametitle{}
	\vfill
	\begin{itemize}
	\item \textit{DSC s kompenzací příkonu} - podstatou DSC s kompenzací příkonu je zachování nulového teplotního rozdílu mezi měřeným a srovnávacím vzorkem. Tato varianta DSC je charakterizována dvěma oddělenými měřícími celami a dvěma tepelnými zdroji a měříme tedy elektrický příkon, který je potřebný k udržení konstantní teploty obou vzorků.
	\item \textit{DSC s tepelným tokem} - druhou variantou je metoda DSC s tepelným tokem. Měření rozdílu příkonu je nahrazeno měřením rozdílu teplot vzorku a srovnávacího vzorku, které jsou umístěny ve společné peci a jsou spojeny tepelným mostem. Se znalostí tepelného odporu mezi pecí a vzorkem a referencí lze považovat tepelný tok od vzorku nebo ke vzorku za úměrný rozdílu teplot. Teplota vzorku je měřena termočlánkem, který je v kontaktu se vzorkem.
	\end{itemize}
	\vfill
}

\frame{
	\frametitle{}
	\vfill
	\begin{figure}
		\adjincludegraphics[height=.7\textheight]{img/STA449C-drzak2.jpg}
		\caption*{DSC držák s Pt/Rh kelímky.}
	\end{figure}
	\vfill
}

\frame{
	\frametitle{}
	\vfill
	\begin{figure}
		\adjincludegraphics[height=.65\textheight]{img/dsc-skalice.png}
		\caption*{DSC křivka modré skalice.}
	\end{figure}
	\vfill
}

\subsection{STA}
\frame{
	\frametitle{}
	\vfill
	\begin{itemize}
	\item Simultánní metody (STA) – umožňují zkoumat více fyzikálních vlastností během jednoho měření. Výhodou tohoto přístupu je, že nemusíme připravovat nové vzorky a máme tak dány stejné experimentální podmínky. Na druhou stranu ale tyto podmínky musí vyhovovat všem použitým metodám.
	\item Mezi nejvíce rozšířenou dvojici metod patří TG-DTA a~TG-DSC. Tyto metody se totiž dobře doplňují.
	\end{itemize}
	\vfill
}

\frame{
	\frametitle{}
	\vfill
	\begin{figure}
		\adjincludegraphics[height=.65\textheight]{img/tg-dsc-skalice.png}
		\caption*{STA modré skalice.}
	\end{figure}
	\vfill
}

\subsection{TMA a dilatometrie}
\frame{
	\frametitle{}
	\vfill
	\begin{itemize}
		\item \textbf{TMA} -- termomechanická analýza, technika používaná ke stanovení rozměrových změn vzorku v závislosti na teplotě, jako je teplota měknutí, teplota skelného přechodu a koeficient lineární teplotní roztažnosti.
		\item \textbf{Dilatometrie} -- měření závislosti rozměrů vzorku na teplotě během definovaného teplotního programu.
		\item \textbf{Tepelná expanze} -- u většiny materiálů dochází ke změně rozměrů během ohřevu nebo chlazení. Během ohřevu se rozměry zpravidla zvětšují a během ochlazování zmenšují.
		\item Změnu rozměrů můžeme popsat pomocí součinitele teplotní délkové roztažnosti, \ce{$\alpha$}.
		\item $l\ =\ l_0(1+\alpha\Delta t)$
	\end{itemize}
	\vfill
}


\subsection{Instrumentace}
\frame{
	\frametitle{}
	\vfill
	\begin{figure}
		\adjincludegraphics[height=.75\textheight]{img/STA449C.jpg}
		\caption*{STA 449C Jupiter.}
	\end{figure}
	\vfill
}

\frame{
	\frametitle{}
	\vfill
	\begin{figure}
		\adjincludegraphics[height=.5\textheight]{img/DTA-\_und\_DSC-Tiegel.jpg}
		\caption*{Kelímky pro TG a DSC.\footnote[frame]{Zdroj: \href{https://commons.wikimedia.org/wiki/File:DTA-_und_DSC-Tiegel.jpg}{Bic/Commons}}}
	\end{figure}
	\vfill
}

\frame{
	\frametitle{}
	\vfill
	\begin{figure}
		\adjincludegraphics[width=.9\textwidth]{img/Autosampler-DSC.jpg}
		\caption*{Autosampler pro TG/DSC.\footnote[frame]{Zdroj: \href{https://commons.wikimedia.org/wiki/File:Autosampler-DSC.jpg}{Dr. Reiner Düren aka RedPiranha/Commons}}}
	\end{figure}
	\vfill
}

\subsection{Coupling TGA/IR}
\frame{
	\frametitle{}
	\vfill
	\begin{itemize}
	\item Plyny vznikající během degradace vzorku vedeme do měřící cely a pomocí IR spektroskopie stanovíme jejich složení
	\item Během transportu plynů z pece do měřící cely dochází k velkému zředění plynu, proto je nutné používat citlivější detektory (MCT)
	\end{itemize}
	\begin{figure}
		\adjincludegraphics[height=.5\textheight]{img/tg-irFoto.jpg}
		\caption*{Coupling TG/IR. Netzsch STA 449C Jupiter-Bruker Tensor 27}
	\end{figure}
	\vfill
}

\frame{
	\frametitle{}
	\vfill
	\begin{figure}
		\adjincludegraphics[height=.75\textheight]{img/tg-ir.png}
		\caption*{Výstup z TG/IR.}
	\end{figure}
	\vfill
}

\subsection{Praktické využití termické analýzy}
\frame{
	\frametitle{}
	\vfill
	\begin{itemize}
	\item Studium degradačních procesů - polymery, organické a anorganické materiály
	\item Charakterizace polymerů - skelný přechod
	\item Farmacie a potravinářství - studium lyofilizačních a krystalizačních procesů
	\item Materiálová chemie kovů - konstrukce fázových diagramů slitin
	\item Geologie - identifikace přírodních materiálů
	\end{itemize}
	\vfill
}

\subsection{Analýza keramických materiálů}
\frame{
	\frametitle{}
	\vfill
	\begin{columns}
	\begin{column}{0.5\textwidth}
	\adjincludegraphics[height=.6\textheight]{img/TG-pottery.png}
	\end{column}

	\begin{column}{0.5\textwidth}
	\adjincludegraphics[height=.6\textheight]{img/TG-pottery-DSC.png}
	\end{column}

	\end{columns}
	\vfill
	Termická analýza jako metoda pro identifikaci keramických materiálů.\footnote[frame]{TSENG, YUNG-KUAN a BI-YAN XU. AN ANALYSIS OF THE GEM-BLUE GLAZE OF YE WANG'S KOJI POTTERY. Archaeometry [online]. 2012, 54(4), 643-663 [cit. 2019-11-22]. DOI: 10.1111/j.1475-4754.2011.00646.x.}
}

\end{document}
